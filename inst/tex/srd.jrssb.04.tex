\def\path{ps/}
\def\texpath{tex/}
\def\additive{additive-2}
\def\variance{variance-2}
\def\nn{nn-2}
\def\incl{\vspace*{2cm}}
\def\incl{\includegraphics}


\documentclass[10pt]{article}
\usepackage{graphicx}
\usepackage{amssymb}
\textwidth=16.6truecm
\textheight=24.8truecm
\hoffset=-2.2truecm
\voffset=-3truecm

\def\1{{\mathbb I}}
\def\const{\hbox{\rm const}}
\def\D{{\,\rm d}}
\def\EE{\mathop{E}\nolimits}
\def\mid{|}
\def\NN{{\mathbb  N}}
\def\RR{{\mathbb  R}}
\def\ZZ{{\mathbb Z}}
\def\bold#1{{\bf #1}}


\begin{document}

\begin{table}
\input \texpath\additive.1.tex 
\end{table}

\begin{table}
\input \texpath\variance.3.tex
\end{table}

\begin{table}
\input \texpath\nn.2.tex
\end{table}


\def\FigOOO#1{
  \label{fig:#1}
   Dependency of the mean rejection rate from the portion $\alpha$
   for the random coin model.
   The Gaussian random field has
  exponential covariance function and scale $s=0.05$;
  the simulations are based on $n=100$ points.
  Left: ``$E(h)=\const$''; right: ``$V(h)=\const$'';
  top: $R=0.05$; bottom: $R=0.1$.
  Each figure gives the performances for all six cross-combinations of 
  tests using the $l_1$ or the $l_\infty$ norm ($\Box$ and $\times$,
  respectively)
  and one of the set of 
  weights $w^{(i)}_3$,   $w^{(i)}_5$, or $w^{(i)}_7$ (dashed line, solid,
  dotted, respectively).
}
 \begin{figure}
 \begin{center}
  \includegraphics[width=6.7cm]{\path\additive 
    .FALSE.TRUE.TRUE.exponential.exponential.5.5.100.E.95.eps}
  \includegraphics[width=6.7cm]{\path\additive 
    .FALSE.TRUE.TRUE.exponential.exponential.5.5.100.V.95.eps}\\[-0mm]
  \includegraphics[width=6.7cm]{\path\additive
    .FALSE.TRUE.TRUE.exponential.exponential.10.5.100.E.95.eps} 
  \includegraphics[width=6.7cm]{\path\additive
    .FALSE.TRUE.TRUE.exponential.exponential.10.5.100.V.95.eps}  
 \caption{\FigOOO{simu.E}}
 \end{center}
\end{figure}


\def\FigOOOO#1{Data sets: (a) Coulissenhieb, the diameters of the
 stems are enlarged by a factor of 5; (b) Gus Pearson,
 the symbols \raise -3pt\hbox{\huge $\cdot$}, $\circ$, 
 {\tiny$\bigtriangleup$}, $+$,
 $\bullet$, 
  represent $0-50$, $50-100$, $100-150$, $150-200$ and $>200 {\;\rm
 cm}^2$ of growth, respectively.\label{fig:#1}}
\begin{figure}
(a) \includegraphics[width=13cm]{\path coulissenhieb.eps}\\
(b) \incl[width=13cm]{\path biondi.etal.eps}
\vspace*{2mm}

\caption{\FigOOOO{data.plot}}
\end{figure}

\def\FigOne#1{Gus Pearson data set: dots: empirical variogram; black curve:
 variogram model; dashed lines: 95\% confidence bounds \label{fig:#1}}
\begin{figure}
 \hfil \incl[height=3.3cm]{\path
   biondi.etal.confidence.gamma.eps}\hfill 
 \caption{\FigOne{confidence}}
\end{figure}

\def\FigO#1{ Characterising functions of the
 data after transformation of the marks to marginally
 Gaussian variables;
 $\bigtriangleup$: Coulissenhieb, $\bullet$: Gus Pearson, 
 (a) $\hat E$, (b) $\hat V$, (c) $\hat \gamma$ and Mat\'ern model
 fitted by maximum likelihood
 \label{fig:#1}}
\begin{figure}
  \hbox{(a) \includegraphics[height=4.4cm]{\path data.E.eps} 
    (b) \includegraphics[height=4.4cm]{\path data.V.eps}
    (c) \includegraphics[height=4.4cm]{\path data.G.eps}}
  \caption{\FigO{data.EVG}}
\end{figure}

% last table missing
\clearpage


\end{document}

%%% Local Variables: 
%%% mode: latex
%%% TeX-master: t
%%% End: 
